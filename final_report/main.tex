%%%%%%%%%%%%%%%%%%%%%%%%%%%%%%%%%%%%%%%%%
%% AWAB QURESHI ASSIGNMENT COMMON FILE %%
%%%%%%%%%%%%%%%%%%%%%%%%%%%%%%%%%%%%%%%%%

% to use, just do \input{path/to/file}

%% Document Basics

% Better document class that can go up to 14pt, 17pt, 20pt
\documentclass[]{extarticle}
% The basic LaTeX/TeX engine expects (or perhaps is meant to process) pure ASCII input,
% whenever you use unicode (non ASCII) characters, you need to use the inputenc package,
% see https://tex.stackexchange.com/questions/370278/is-there-any-reason-to-use-inputenc
\usepackage[utf8]{inputenx}
% Set the margin away from that terrible built-in margin
% \usepackage[margin=1in]{geometry}
\usepackage{fullpage}

%% Formatting Packages

\usepackage{lmodern}
% The default fontencoding is only 7-bit, this sets it to 8-bit,
% see https://tex.stackexchange.com/questions/664/why-should-i-use-usepackaget1fontenc
\usepackage[T1]{fontenc}

% The full ams set includes so many symbols that you need 
% (you don't need amsmath with mathtools but eh)
\usepackage{amsmath, amsfonts, amssymb, amsthm, mathtools}

% More symbols, use \mathscr{B} for example
\usepackage{mathrsfs}

% Lets you write inline codestyle with \verb|a|
\usepackage{verbatim}

% Symbols of Bra-ket Vectors of the form <a|b> and <a>
\usepackage{braket}

% The parskip package helps in implementing paragraph layouts where the paragraphs are separated by a vertical space instead of (or in addition to) indenting them.
\usepackage{parskip}

% Used for long urls to stay within the margin
\usepackage{xurl}

% Set the header on the top of every page
\usepackage{fancyhdr}

% Multicolumn Enviroments to split up pages
\usepackage{multicol}

% To add in pictures
\usepackage{graphicx}

% To properly align systems of equations
\usepackage{systeme}

% "provides non expandable macros and tests operating on "strings of tokens""
% see https://texdoc.org/serve/xstring/0
\usepackage{xstring}

% Used to draw diagonal lines (“cancelling” a term) and arrows with limits
\usepackage{cancel}

% To draw cool diagrams
\usepackage{tikz}
% To draw trees easier
\usepackage{forest}

% Far more flexible when compared with enumerate
% the option should make labels take up less horizontal space
\usepackage[shortlabels]{enumitem}

% To render out pseudocode well
% https://tug.ctan.org/macros/latex/contrib/algorithm2e/doc/algorithm2e.pdf
\usepackage[ruled,vlined,linesnumbered]{algorithm2e}

% utility commands to turn on and off algorithm numbering
\makeatletter
\newcommand{\RemoveAlgoNumber}{\renewcommand{\fnum@algocf}{\AlCapSty{\AlCapFnt\algorithmcfname}}}
\newcommand{\RevertAlgoNumber}{\algocf@resetfnum}
\makeatother

\RemoveAlgoNumber % please don't number the algorithms, I'd rather do that myself with \label{} and \ref{} with \caption{}
\DontPrintSemicolon % \; marks the end of a line but don't print it
% Also note that \BlankLine prints a blank line
\SetKwProg{Fn}{}{ : }{} % function with Fn should have no prefix nor suffix % in order to display function name correctly, run \SetKwFunction{nameCmd}{funcName} to make \nameCmd generate a syntax highlighted function with name funcName
\SetKwInOut{Input}{Input} % input and output
\SetKwInOut{Output}{Output}
\SetKw{None}{None} % make \None a keyword
\SetKw{True}{True} % make \True a keyword
\SetKw{False}{False} % make \False a keyword
\SetNoFillComment % comment with \tcc will look like /* aslkdja */ instead of placing */ at very end

% To render and syntax highlight real code well too
\usepackage{listings}
\definecolor{mauve}{rgb}{0.58,0,0.82}
\lstset { %
	frame=tb,
	language=[11]C++,
	aboveskip=3mm,
	belowskip=0mm,
	showstringspaces=false,
	columns=flexible,
	basicstyle={\ttfamily},
	numbers=none,
	numberstyle=\tiny\color{gray},
	keywordstyle=\bfseries,
	commentstyle=\itshape\color{gray},
	stringstyle=\itshape\color{darkgray},
	directivestyle=\bfseries\itshape\color{darkgray},
	breaklines=true,
	breakatwhitespace=true,
	tabsize=2
}

% Get rid of >> and << turning into ugly ligatures in listings
\usepackage{microtype}
\DisableLigatures[>,<]{encoding = T1,family=tt*}

% colour time
\usepackage{xcolor}

% define darkmode colours
\definecolor{bgcolour}{rgb}{0.11,0.11,0.11} %Background : almost black
\definecolor{txtcolour}{rgb}{0.9,0.9,0.9} %Font : an off-white

% set page colours
% \pagecolor{bgcolour}
% \color{txtcolour}

% Change margin for different parts of document
\usepackage{changepage}
%\begin{adjustwidth}{left amount}{right amount}
% \lipsum[2]
% \end{adjustwidth}
% https://tex.stackexchange.com/questions/588/how-can-i-change-the-margins-for-only-part-of-the-text

\setlength{\columnseprule}{0.5pt}
\def\columnseprulecolor{\color{white}}

% Stop indenting paragraphs please
% \setlength{\parindent}{0pt}
% UPDATE: This might be done by parskip package instead now

% LaTeX is a bit cramped, and I like 1.2 spacing.
% At least it's not double spaced
\usepackage{setspace}
\setstretch{1.2}

% Make links clickable but don't colour them
% Note: You should load setspace before hyperref
\usepackage[hidelinks]{hyperref}

\pagenumbering{gobble}

\theoremstyle{remark}
\newtheorem{prp}{Proposition}

\newtheorem{claim}{Claim}

%% CUSTOM COMMANDS

% Much easier equation alignment environment
% \newcommand{\e}[1]{
% 	\begin{equation*}
% 		\begin{split}
% 			#1
% 		\end{split}
% 	\end{equation*}
% }
\newcommand{\e}[1]{
	\begin{align*}
			#1
	\end{align*}
}

% Piecewise functions!
\newcommand{\pw}[1]{
	\begin{cases}
		#1
	\end{cases}
}

% The sign of a number?
\newcommand{\sign}{
	\text{  sign }
}

% Span of a vector
\newcommand{\Span}[1]{\text{span}(#1)}

% Much nicer way to have auto resizing brackets
\newcommand{\br}[1]{
	\left( #1 \right)
}

% Align systems of equations
\newcommand{\system}[1]{
	% remember to use \+ and \- to escape them if you dont want to insert them in a term
	% eg:
	% (2 \+ a) x + (3 \+ b) y = 0
	% \-a x + (\-2 \+ b) y = 0
	% systeme splits the terms on a + or a -
	% also have https://ctan.mirror.rafal.ca/macros/generic/systeme/systeme_fr.pdf 
	\begin{equation*}
		\sysalign{c,c} \systeme{#1}
	\end{equation*}
}

% Proof enviroment I don't use
\newcommand{\prf}[1]{
	\begin{proof}
		#1
	\end{proof}
}

% Set operations
\newcommand{\U}{\cup}
\newcommand{\n}{\cap}

% Thick characters
\newcommand{\F}{\mathbb{F}}
\newcommand{\R}{\mathbb{R}}
\newcommand{\N}{\mathbb{N}}
\newcommand{\Z}{\mathbb{Z}}
\newcommand{\C}{\mathbb{C}}
\newcommand{\Q}{\mathbb{Q}}

% partial derrivatives
\newcommand{\del}{\partial}
\newcommand{\pd}[2]{\frac{\del #1}{\del #2}}
\newcommand{\hess}{\text{Hess }}

% Divides and doesn't divide
\newcommand{\divs}{\mid}
\newcommand{\ndivs}{\nmid}

% Varphi useful
\newcommand{\vp}{\varphi}

% Something mod something
% although \pmod is pretty good too
\newcommand{\Mod}[1]{\ (\mathrm{mod}\ #1)}

% This is where that Bra-Ket package comes in
\newcommand{\pideal}[1]{\langle #1 \rangle}

% Trace of a tensor
\newcommand{\trace}{\text{trace}}

% Rank of a tensor
\newcommand{\rank}{\text{rank}}

% Image of a map
\newcommand{\image}{\text{Im }}

% Cute little star at the right of the page
\newcommand{\Qed}{
	\begin{flushright}
		$\large ★\bigstar$
	\end{flushright}}

% Why is default epsilon the symbol for set include
\renewcommand{\epsilon}{\varepsilon}

% Floor and cieling
\newcommand{\floor}[1]{\lfloor #1 \rfloor}
\newcommand{\ceil}[1]{\lceil #1 \rceil}

% Pictures centered vertically in line
\newcommand{\vcenteredinclude}[2]{\begingroup
	\setbox0=\hbox{\includegraphics[#1]{#2}}%
	\parb}

% Norm ||x|| for topology
\newcommand{\norm}[1]{\left\lVert#1\right\rVert}

% Add scoped colour to math text with correct spacing
% From https://tex.stackexchange.com/questions/21598/how-to-color-math-symbols
\makeatletter % make the @ symbol just another letter
\def\mathcolour#1#{\@mathcolour{#1}} % helper function
\def\@mathcolour#1#2#3{% 
	\protect\leavevmode
	\begingroup
	\color#1{#2}#3 % add colour to text in group environment for correct spacing
	\endgroup
}
\makeatother

% Create a nice large red box that screams "YOU LEFT THIS TODO"
% From https://tex.stackexchange.com/questions/208128/how-to-make-an-empty-box
\newcommand{\TODO}{\mathcolor{red}{
		\raisebox{-5.5\unitlength}{
			\framebox(20,20){}}
	}}

% Title (what no way)
% \title{Assignment}
% \author{awab qureshi}
% \date{October 2022}

% This lets you set headers over your pages
% See more https://tex.stackexchange.com/questions/111320/how-to-print-the-section-name-without-the-section-number-in-a-fancyhdr-header
% \pagestyle{fancy}

% so that fancyhead stops warning me about too little space
% \setlength{\headheight}{15pt}

%% This style sucked
% \fancyhead{}% clear
% \fancyhead[L]{{Awab Q.}}
% \fancyhead[R]{\rightmark}

% Nice header with section name and person name
% \renewcommand{\sectionmark}[1]{\markright{#1}}
% \fancyhf{}
% \rhead{\fancyplain{}{Awab Q.}}
% \lhead{\fancyplain{}{\rightmark}}

% Remove the section numbering please
% Better than using \section*
\usepackage{titlesec}

% Place each section on a new page.
% Seems weird but assignments often need each question on a page by itself
% \AddToHook{cmd/section/before}{\clearpage}

%%%%%%%%%%%%%%%%%%%
%%  EXAMPLE USE  %%
%%%%%%%%%%%%%%%%%%%
%
% \input{path/to/common}
%
% \begin{document}
%
% % \maketitle
%
% \section{a}
%
% \pagebreak
% aaslkdj
% \pagebreak
%
% \section{b}
%
% \pagebreak
%
% \end{document}
%
%%%%%%%%%%%%%%%%%%


\usepackage[
    backend=biber,
    style=ieee,
]{biblatex}

\addbibresource{bibliography.bib}

\title{CS 488 Proposal}
\author{Palaksha Drolia, Awab Qureshi}
\date{25 June 2025}

\begin{document}

~\vfill
\begin{center}
\Large

CS 488 Project Report

\Huge 
Water Physics with Rigid Bodies
\vspace{0.5em}

\large
\textit{Formerly, Bottles \& Water}
\vspace{1em}

\textit{Palaksha Drolia, Awab Qureshi}

\textit{pdrolia, a9quresh}

\end{center}
\vfill ~\vfill~
\newpage

\section{Final Report}
\subsection{Rasterisation and Rendering pipeline}
We used OpenGL for rasterisation to reduce CPU load and utilise GPU more effectively. 
Drawing triangles to the screen was done using the standard process described in \cite{LearnOpenGL:Ch5}.
We integrated the transformation matrix and our shade function into the OpenGL pipeline to form our vertex and fragment shader, respectively. This was written in GLSL and utilised uniforms passed by our program for effective rendering. We also created a Phong-based shader that accounts for ambient, diffuse, and specular contributions to give a more realistic output.
We also created a high-level object-oriented wrapper over certain OpenGL functionality, like shaders, to easily set and verify uniforms.
\subsection{Shallow-Water Simulation}
The main focus of our project is the physics-based simulation of a shallow, large body of water: the pool.
In order to maintain a level of interactability, we use a two-dimensional height field for simulation.

For this, we employ the shallow water equations as given in \cite{hfluid},
\begin{align*}
    \frac{Dh}{Dt} = -h (\nabla \cdot v) \hspace{2em} \frac{Dv}{Dt} = -g\nabla \eta + a^{ext}
\end{align*}
where $h$ is the depth of the water, $H$ is the $y$-coordinate of the terrain on the bottom, $\eta = H+h$ is the $y$-coordinate of the water's surface, $v$ is the vector $(u,w)$ representing the horizontal velocity of the fluid, $g$ is gravity, $a^{ext}$ is the external acceleration, and $D$ is the material derivative operator as given in the course slides \cite{lec:waves}.

For better accuracy, we implement a staggered grid for our discretised simulation as specified briefly in the course slides \cite{lec:waves} and \cite{hfluid}.
We store the heights of a cell, $h_{i,j}$ and $H_{i,j}$ at its centre.
And we store the velocity components $u_{i+\frac12, j}, w_{i, j+\frac12}$ on the faces.
When computing values not stored, we bilinearly interpolate.

% velocity advection section here
The grid is first updated with values generated by computing the advection of height and velocity.
We use the semi-lagrangian method as proposed in \cite{lec:waves} in order to solve the advection of $h_{i,j}$, $u_{i+\frac12, j}$ and $v_{i, j+\frac12}$.
Let $x_u = ((i+\frac12) \Delta x, j \Delta x)$ be the position of the grid cell at $u_{i+\frac12, j}$ and similarly define $x_v$ and $x_h$
We thus compute the new values,
    $u^{n+1}_{i+\frac12, j} = interp(x_u - \Delta t \cdot (u_{i+\frac12,j}, w_{i,j})^T)$,
    and $w^{n+1}_{i, j+\frac12} = interp(x_v - \Delta t \cdot (u_{i,j}, w_{i,j+\frac12})^T)$,
    and the height as $h^{n+1}_{i,j} = x_h - \Delta t \cdot (i \Delta x, j \Delta x)$.

% velocity and height update section here
The heights are integrated by adding the following,
$$h_{i,j} = - \br{(\overline{h}_{i+\frac12, j} u_{i+\frac12, j} - \overline{h}_{i-\frac12, j} u_{i-\frac12, j})/{(\Delta x)} +  (\overline{h}_{i, j+\frac12} w_{i, j+\frac12} - \overline{h}_{i, j-\frac12} w_{i, j-\frac12})/({\Delta x}) } \Delta t$$
as suggested by \cite{hfluid}, and \cite{lec:waves}.
Here we implement a heuristic provided by \cite{hfluid}, where instead of linearly interpolating to find the values of $\overline h_{\_, \_}$, we instead evaluate it to be equal to $h$ in the upwind direction.

The velocities are updated, as \cite{hfluid} suggests, taking the gradient of the water height.
For our staggered velocities, we add the following,
\begin{align*}
    u_{i+\frac12, j} \mathrel{+}= \br{\frac{-g}{\Delta x} (\eta_{i+1, j} - \eta_{i,j}) + a^{ext}_x} \Delta t 
    \hspace{2em}
    w_{i, j+\frac12} \mathrel{+}= \br{\frac{-g}{\Delta x} (\eta_{i, j+1} - \eta_{i,j}) + a^{ext}_z} \Delta t 
\end{align*}
where $a^{ext}$ is the external acceleration.

For boundary conditions, our pool has a well-defined boundary that reflects the waves. We carry out the method suggested in \cite{lec:waves} and at the boundary set the heights to the same as their neighbours, and set the velocity component into the wall to be zero.

\subsection{Rasterising Water Surface}

We rasterise the water surface with the method described in \cite{hfluid}.
The height field of our fluid is already a gird of quads, with each $(i,j)$ vertex having height $\eta_{i,j}$ that we may split into triangles.
We copy over the vertex data to the GPU and split the quads to emit triangles in a geometry shader.
Normals for the water surface are computed on the CPU.

Additionally, as suggested by \cite{hfluid}, we slice the quad into triangles across the following diagonal,
$$
\begin{cases}
    (i,j) - (i+1, j+1) & \text{if }\eta_{i,j} + \eta_{i+1,j+1} > \eta_{i+1,j}+\eta_{i,j+1} \\
    (i+1, j) - (i, j+1) & \text{else}
\end{cases}
$$
since picking the diagonal that aligns with the wave's crests reduces artefacts.

For transparency, we rely on OpenGL's blending.
After enabling OpenGL's blending, we draw the opaque pool first and then draw the transparent water surface as suggested in \cite{LearnOpenGL:Ch24}.

\subsection{Rigid-body Physics}

For rigid-body physics, we employ Verlet integration as suggested by \cite{lec:particles}.
A force acts on the centre of mass of the rigid body whose next coordinate is computed via 
$x_{n+1} \approx x_n + (x_n - x_{n-1}) + \Delta t^2 F_n/m$
where $x_n$ is the vector that represents the coordinates of the body, $m$ is its mass, $\Delta t$ is the time step, and $F_n$ is the force acting on it. Additionally, we also apply gravity to the body.

For torque and rotation, we calculate the updated angular momentum as $L_{n+1} \approx L_n + \Delta t \tau$ where $L_n$ is the angular momentum at time point $n$, $\Delta t$ is the time step, and $\tau$ is the torque acting on it.

\paragraph{Calculating Moment of Inertia}
Next we need to find the Moment of Inertia for our rigid body to compute angular velocity and the new rotation. We achieve this by iterating through the triangular faces in our mesh. Note that we only need to calculate the initial inertia tensor as we can obtain the global inertia tensor at any point of time via
$I = RI_0R^T$
where $I_0$ is the initial rotation tensor and $R$ is the current rotation of the object as a rotation matrix. As suggested by \cite{inertia:polyhedral}, we need to compute 10 integrals in order to compute the volume, centre of mass, and initial inertia tensor. As mentioned in the paper, we can simplify these 10 integrals over the volume using the divergence theorem that states:
$$\int_V \nabla\cdot{\mathbf{F}}\ dV = \int_{\partial V} \mathbf{F} \cdot \mathbf{\hat{n}}\ dA$$
Using this, we appropriate $\textbf{F}$ to convert the volume integrals into corresponding area integrals. Since they are area integrals, we can sum them over each individual face.

We solve these integrals using Barycentric parametrisation as suggested by \cite{inertia:triangular} to get a closed-form result. Using these values, we can fill in the entries in the inertia tensor matrix.

Now, with the angular momentum and the current inertia tensor, we can calculate the angular velocity as
$\omega = I^{-1}L$.
We use a forward Euler scheme to update the rotation quaternion along the angular velocity's axis with the magnitude of $\|\omega\|\Delta t$.

For detecting collisions with objects other than the water's surface, we use a simple axis-aligned bounding box around the objects.
We resolve collisions by finding the axis of minimum overlap, and correct based on that as suggested in \cite{aabb_collision}. 
We also calculate the force that would have acted on the objects and apply torque based on that. We also use this approach to ensure that objects stay inside the pool.
\subsection{Rigid-body and Water Interaction}
When a rigid body lands in the pool, we wish to modify the height and velocity of the fluid.
To accomplish this, we utilise Algorithm 2 as given in \cite{hfluid}.
We essentially subdivide the rigid body's triangles into smaller triangles until their area falls below $\Delta x^2$ where $\Delta x$ is our grid spacing recursively.
This subdivision allows fine-grained spatial resolution for capturing interactions accurately between the body and the fluid.
For each of these small triangles, the centroid's position $p = (p_x, p_y, p_z)$ and its velocity $v = (v_x, v_y, v_z)$ is computed via barycentric interpolation.
We use the triangle's normal $n$ to determine the direction and magnitude of fluid displacement.

The velocity's magnitude relative to the vertical direction is then used to determine how many substeps to divide the current simulation timestep into,
$$num\_substeps = \max\left\{1, \left\lfloor |v- v_y y| (\Delta t/\Delta x) + 0.5 \right\rfloor\right\}$$
This is because more substeps allow smoother and more stable application of forces along the triangle’s trajectory.
For each substep, then, we advance the centroid position along its velocity, identify the fluid grid cell closest to this position, $(i,j)$ and calculate the depth of the centroid relative to the fluid surface height.
Upon finding the centroid submerged, we compute a decay factor that exponentially reduces influence with depth, reflecting that deeper submerged parts affect the fluid less.
The fluid surface's height at the grid cell is then updated by adding a volume displacement proportional to the triangle’s area, velocity, and decay factor,
$$h_{i,j} \mathrel{+}= e^{-(\eta_{i,j}-p_y)} \frac{(n \cdot v_{rel}) A \Delta t}{num\_substeps(\Delta x)^2}$$
where $v_{rel} = v - v_{fluid}$ and $depth=\eta_{i,j}-p_y$, $V_{disp} =n \cdot v_{rel}$.
The fluid surface's velocity is then updated at the corresponding staggered grid faces by pushing them towards the triangle centroid's velocity, scaled by a coefficient,
$$\text{\textit{coeff}} = \min\left\{1, \frac{e^{-(\eta_{i,j}-p_y)}}5 \frac{(\eta_{i,j}-p_y)}{\eta_{i,j}} sign \frac{\Delta t}{(\Delta x)^2} A \right\}$$

where $sign$ is the sign of $\eta_y$.

The fluid must also interact with the rigid body.
For this we consider buoyancy, drag and lift forces.
We compute the sum of these forces $F_i = f_{buoyancy}, f_{drag}, f_{list}$ for each centroid of our subdivided triangle grid from before. We compute these forces by the very straightforward equations 14, 15, 16 and 17 as given in \cite{hfluid} which we do not copy here for brevity. The equations let us adjust coefficients $C_D$, $C_L$ and $\omega$ for the drag, lift, and effective area respectively. Based on our research and experiments, we decided the values $C_D = 0.82$, $C_L = 0.007$, and $\omega = 0.9$. These are based on values for typical cylindrical objects.

The total force acting on the body at its centre of mass is given by $F = \sum F_i$.
And each $F_i$ produces a torque about the body's centre of mass, so the total torque generated is $\tau = \sum r_i \times F_i$, where $r_i$ the displacement from $p$ to the centre of mass, which is used to update the angular velocity of the body.

\newpage

\section{Bibliography}
\printbibliography[heading=none]
    
\end{document}
